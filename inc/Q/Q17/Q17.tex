\begin{figure}[ht!]
	\centering
	\begin{Karnaugh}{$Q^B_n \cdot Q^C_n$}{$X \cdot Q^A_n$}
		\contingut{0,0,0,1,0,x,x,x,0,0,0,0,0,x,x,x}
		\implicant{3}{7}{red}
	\end{Karnaugh}
	\caption{Karnaugh map for the Input of flip-flop $D_A$}\label{fig:input_D_A_2}
\end{figure}\FloatBarrier

\begin{equation}\label{eq:input_D_A_2}
	D_A = \overline{X} \cdot Q^B_n \cdot Q^C_n
\end{equation}

Figure~\ref{fig:input_D_A_2} shows the Karnaugh map for the input of flip-flop $D_A$. The Karnaugh map was created using the state transition table shown in Table~\ref{tab:state_transition_table_4}. The input of flip-flop $D_A$ is given by Eq.~\ref{eq:input_D_A_2}.

\begin{figure}[ht!]
	\centering
	\begin{Karnaugh}{$Q^B_n \cdot Q^C_n$}{$X \cdot Q^A_n$}
		\contingut{0,1,1,0,0,x,x,x,0,0,0,0,0,x,x,x}
		\implicant{2}{6}{red}
		\implicant{1}{5}{red}
	\end{Karnaugh}
	\caption{Karnaugh map for the Input of flip-flop $D_B$}\label{fig:input_D_B_2}
\end{figure}\FloatBarrier

\begin{equation}
	\begin{split}
		D_B &= \overline{X} \cdot \overline{Q^B_n} \cdot Q^C_n + \overline{X} \cdot Q^B_n \cdot \overline{Q^C_n}\\
		&= \overline{X} \left( \overline{Q^B_n} \cdot Q^C_n + \cdot Q^B_n \cdot \overline{Q^C_n} \right)\\
		&= \overline{X} \left( Q^B_n \oplus Q^C_n \right)
	\end{split}
	\label{eq:input_D_B_2}
\end{equation}

Figure~\ref{fig:input_D_B_2} shows the Karnaugh map for the input of flip-flop $D_B$. The Karnaugh map was created using the state transition table shown in Table~\ref{tab:state_transition_table_4}. The input of flip-flop $D_B$ is given by Eq.~\ref{eq:input_D_B_2}.

\begin{figure}[ht!]
	\centering
	\begin{Karnaugh}{$Q^B_n \cdot Q^C_n$}{$X \cdot Q^A_n$}
		\contingut{0,0,1,0,0,x,x,x,1,1,1,1,1,x,x,x}
		\implicant{12}{10}{red}
		\implicant{2}{10}{red}
	\end{Karnaugh}
	\caption{Karnaugh map for the Input of flip-flop $D_C$}\label{fig:input_D_C_2}
\end{figure}\FloatBarrier

\begin{equation}
	\begin{split}
		D_C &= X + Q^B_n \cdot \overline{Q^C_n}\\
	\end{split}
	\label{eq:input_D_C_2}
\end{equation}

Figure~\ref{fig:input_D_C_2} shows the Karnaugh map for the input of flip-flop $D_C$. The Karnaugh map was created using the state transition table shown in Table~\ref{tab:state_transition_table_4}. The input of flip-flop $D_C$ is given by Eq.~\ref{eq:input_D_C_2}.

\begin{figure}[ht!]
	\centering
	\begin{Karnaugh}{$Q^B_n \cdot Q^C_n$}{$X \cdot Q^A_n$}
		\contingut{0,0,0,0,0,x,x,x,0,0,0,0,1,x,x,x}
		\implicant{12}{14}{red}
	\end{Karnaugh}
	\caption{Karnaugh map for the Output $Z$}\label{fig:Z}
\end{figure}\FloatBarrier

\begin{equation}
    Z = X \cdot Q^A_n 
    \label{eq:Z}
\end{equation}

Figure~\ref{fig:Z} shows the Karnaugh map for the output $Z$. The Karnaugh map was created using the state transition table shown in Table~\ref{tab:state_transition_table_4}. The output $Z$ is given by Eq.~\ref{eq:Z}.
